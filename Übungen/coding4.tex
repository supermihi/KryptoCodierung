% load document class mhexsheet & select language as option (german or english)
\documentclass[german]{mhexsheet}

% set the parameters for the current exercise sheet with \exerciseSetup (key-value interface)
\exerciseSetup{
  lecture      = Einführung in die Informations- und Codierungstheorie,
  lectureshort = Codierungstheorie, % optional short name for the footer
  semester     = SS 2015,
  deadline     = {29.6.2015},
  lecturer     = Michael Helmling,
  operator     = Carolin Torchiani,
  homepage     = {http://userpages.uni-koblenz.de/~helmling/coding},
  sheetnumber  = 10,
  %solution, % solution (or solution=true) enables output of solution environments. solution=false (default) disables it,
  % logo = \includegraphics[width=5cm]{image.png}, % can override the standard logo
  % logowidth = 7cm % ... and also its width
}
\usepackage{cleveref}
\usepackage[
  binary-units=true,
  per-mode=symbol,
  exponent-product=⋅,
  locale=DE]{siunitx}
\usepackage{amssymb}
\usepackage[math,fonts=false]{mh_basic}
% == local commands 
\setcounter{MaxMatrixCols}{20}
\newcommand\A{\mathcal A}
\newcommand\B{\operatorname{B}}
\renewcommand\d{\operatorname{d}}
\DeclareMathOperator{\w}{w}
\newcommand{\Mat}{\operatorname{Mat}}
\renewcommand{\t}[1]{\texttt{#1}}
\renewcommand{\(}{\left(}
\renewcommand{\)}{\right)}
\renewcommand\P{\mathcal P}
\newcommand\pErr{p_\mathrm{Err}}
\newcommand\D{D_{\mathrm{KL}}}

\newcommand{\length}{l}
\newcommand{\mc}{\mathcal}
\newcommand{\mb}{\mathbf}
\newcommand{\ol}[1]{[#1]}
\newcommand{\ov}[1]{\overline{#1}}
\newcommand{\id}{\operatorname{id}}
\newcommand{\Sy}{\mathbb S}
\newcommand{\GF}{\operatorname{GF}}
\newcommand{\Ce}{\mathbb C}
\newcommand{\Primes}{\mathbb P}
\newcommand{\Nk}{\operatorname {Nk}}
\newcommand{\Nr}{\operatorname {Nr}}
\newcommand{\AES}{\mathrm{AES}}
\newcommand{\ML}{\mathrm{ML}}
\newcommand{\expand}{\texttt{KeyExpansion}}
\newcommand{\size}{\operatorname{size}}
\newcommand{\parf}{\operatorname{par}}
\newcommand{\im}{\operatorname{im}}
\newcommand{\K}{\mathcal K}
\DeclareMathOperator\Pot{Pot}
\usepackage{tikz}
\renewcommand\C{\mathcal C}
\tikzset{
  extend/.style={shorten >=-#1,shorten <=-#1},
  extend/.default=5mm,
  shorten/.style={shorten >=#1,shorten <=#1},
  shorten/.default=5mm,
  wobbly/.style={decorate,decoration={snake,segment length=15pt,amplitude=0.5pt}},
  brace/.style={decorate,decoration={brace,#1}},
  box/.style={draw,fill=#1,rounded corners,minimum height=15pt},
  circ/.style={circle,inner sep=0mm,minimum size=2mm,draw},
  dot/.style={circle,inner sep=0mm,minimum size=2mm,fill}
}
% ==================

\setmathfont{texgyretermes-math.otf}
\setmathfont[range={"29F5}]{XITS Math}
\setmathfont[range={}]{texgyretermes-math.otf}
\usepackage{booktabs}
\newcommand{\mc}{\mathcal}
\newcommand{\GF}{\operatorname{GF}}
\begin{document}
\maketitle

\begin{exercise}
  Im Abschnitt \enquote{Blockcodes} werden $K$-Wörter aus der binären Quelle, die wir uns als Ausgang einer Quellencodierung vorgestellt haben, zu $N$-Wörtern codiert. Begründen Sie, warum jedes dieser $K$-Wörter mit gleicher Wahrscheinlichkeit vorkommt, wenn eine \enquote{gute} Quellencodierung gewählt wurde.
\end{exercise}

\begin{exercise}
  Zeigen Sie: Die Kapazität des BEC($ε$) ist $1-ε$.
\end{exercise}

\begin{exercise}
  \begin{Center}
  \begin{tikzpicture}[scale=.75]
        \node (x0) at (0, 0) {$0$};
        \node (x1) at (0,-2) {$1$};
        \node (y0) at (2, 0) {$0$} edge[<-] node[above] {$1$} (x0)
                                   edge[<-] node[below] {$ε$}   (x1);
        \node (y1) at (2,-2) {$1$} edge[<-] node[below] {$1-ε$} (x1);
      \end{tikzpicture}
   \end{Center}
  Der \emph{Z-Kanal} Z($ε$) mit Ausgabealphabet $\GF(2)$ ist definiert über
  \[ P_{Z(ε), 0}(y) = \begin{cases} 1&\text{falls }y=0,\\0&\text{sonst}\end{cases} \quad\text{und}\quad P_{Z(ε), 1}(y) = \begin{cases} ε&\text{für } y=0,\\1-ε&\text{sonst.}\end{cases}    \]

  Sei $ε=\num{.15}$ und $P_X(0)=\num{.9}$.
  \begin{enumerate}
    \item Berechnen Sie $P_{X∣Y=1}(1)$ und $P_{X∣Y=0}(1)$!
    \item Berechnen Sie $I(X∥Y)$ und vergleichen Sie das Ergebnis mit dem in Beispiel~5.39 ermittelten Wert für den BSC(\num{.15}) bei derselben Eingangsverteilung.
    \item Berechnen Sie die Kapazität und die optimale Inputverteilung von Z(\num{.15}) mit Hilfe eines Computers.
  \end{enumerate}
\end{exercise}

\begin{exercise}
  Sei $X$ eine Zufallsvariable mit Alphabet $A⊆ℝ$ und Erwartungswert $μ_X=\E[X]$. Zeigen Sie:
  \begin{enumerate}
    \item Es gibt mindestens ein $x∈A$ mit $x≤μ_X$.
    \item Ist $X$ gleichverteilt, $A⊆ℝ_0^+$ und $\abs A$ gerade, gilt $x≤2μ_X$ für mindestens die Hälfte aller $x∈A$.
  \end{enumerate}
\end{exercise}



 
\end{document}