% load document class mhexsheet & select language as option (german or english)
\documentclass[german]{mhexsheet}

% set the parameters for the current exercise sheet with \exerciseSetup (key-value interface)
\exerciseSetup{
  lecture      = Einführung in die Kryptographie,
  lectureshort = Kryptographie, % optional short name for the footer
  semester     = SS 2015,
  deadline     = {11.05.2015, 12:00 Uhr},
  date  = {12.05.2015},
  lecturer     = Carolin Torchiani,
  operator     = Michael Helmling,
  %homepage     = www.google.de,
  sheetnumber  = 4,
  %solution, % solution (or solution=true) enables output of solution environments. solution=false (default) disables it,
  % logo = \includegraphics[width=5cm]{image.png}, % can override the standard logo
  % logowidth = 7cm % ... and also its width
}



\usepackage[math,fonts=false]{mh_basic}
\setmathfont{texgyretermes-math.otf}
\setmathfont[range={"29F5}]{XITS Math}
\setmathfont[range={}]{texgyretermes-math.otf}

\newcommand{\mc}{\mathcal}

\begin{document}
\maketitle

\begin{exercise}[title = Diffusion affin linearer Blockchiffren]
 Konzipieren Sie eine Verschlüsselungsfunktion einer affin linearen Blockchiffre der Blocklänge $4$ auf $\Z_3$, die folgende Eigenschaft erfüllt:
 \begin{enumerate}
  \item Der erste Buchstabe des Klartexts $p = (p_1, \dots, p_4) \in \Z_3^4$ beeinflusst genau den dritten Buchstaben des Chiffretexts. Anders ausgedrückt: Ändert sich nur (!) der erste Buchstabe des Klartexts, ändert sich nur (!) der dritte Buchstabe des Chiffretexts.
  \item Der erste Buchstabe des Klartexts $p = (p_1, \dots, p_4) \in \Z_3^4$ beeinflusst jeden Buchstaben des Chiffretexts.
  \item Jeder Buchstabe des Klartexts $p = (p_1, \dots, p_4) \in \Z_3^4$ beeinflusst jeden Buchstaben des Chiffretexts.
 \end{enumerate}
\end{exercise}


\begin{exercise}[title = Beispiele affin linearer Blockchiffren]
\begin{enumerate}
 \item Modellieren Sie die Vigenère-Chiffre mit fester Schlüssellänge $n \in \N$ als affin lineare Blockchiffre. 
 \item Zeigen Sie, dass Permutationschiffren lineare Blockchiffren sind.
 \end{enumerate}
\end{exercise}

\begin{exercise}[title = Knacken affin linearer Blockchiffren]
 \begin{enumerate}
  \item Das Alphabet einer linearen Blockchiffre mit Blocklänge $2$ sei $\{A, \dotsc, Z\}$. Es sei außerdem bekannt, dass HAND in FOOT verschlüsselt und der ECB-Mode genutzt wird. Bestimmen Sie die Schlüsselmatrix $A \in \Z_{26}^{2 \times 2}$.
  \item Implementieren sie einen Algorithmus zum Knacken affin linearer Blockchiffren durch einen Known-Plaintext-Angriff.
 \end{enumerate}
\end{exercise}


\begin{exercise}[title = Verschlüsselungsmodi]
 \begin{enumerate}
  \item Wir wählen den Schlüssel $e = ([1], [0], [1], [1]) \in \Z_2^4$, außerdem $c_{0} = ([0], [0], [1], [1]) \in \Z_2^4$ und den Klartext
  \[p = ([0], [1], [0], [0], \; [0], [1], [0], [0], \; [1], [1], [1], [0],\; [1], [0], [1], [1]) \in \Z_2^{16}.\]
  Verschlüsseln Sie $p$ mit den Verschlüsselungsmodi ECB-, CBC-, CFB- und OFB-Mode unter Nutzung des Vernam-One-Time-Pads als zugrunde liegender Blockchiffre. 
 
  Nehmen Sie nun an, dass 
  \[c = ([1], [1], [0], [1], \; [1], [1], [0], [1], \; [0], [0], [1], [0],\; [1], [0], [0], [1]) \in \Z_2^{16}\]
  mit $e$ und dem ECB-, CBC-, CFB- bzw. OFB-Mode verschlüsselt wurde, wieder mit dem Vernam-One-Time-Pad als Blockchiffre. Bestimmen Sie jeweils den Klartext.  
%   \item Beschreiben Sie die iterativen Verfahren, mit denen die Verschlüsselungsmodi ECB-, CBC-, CFB- und OFB-Mode entschlüsselt werden können. Gehen Sie davon aus, dass der Entschlüsselungsschlüssel $d \in \mc K$ der zugrunde liegenden Blockchiffre bekannt ist.
  \item Implementieren Sie die Verschlüsselungsmodi ECB, CBC, CFB und OFB. Nutzen Sie als zugrunde liegende Blockchiffre das bereits implementierte Vernam-One-Time-Pad.
 \end{enumerate}
\end{exercise}

\end{document}