% load document class mhexsheet & select language as option (german or english)
\documentclass[german]{mhexsheet}

% set the parameters for the current exercise sheet with \exerciseSetup (key-value interface)
\exerciseSetup{
  lecture      = Einführung in die Kryptographie,
  lectureshort = Kryptographie, % optional short name for the footer
  semester     = SS 2015,
  deadline     = {01.06.2015, 12:00 Uhr},
  date  = {02.06.2015},
  lecturer     = Carolin Torchiani,
  operator     = Michael Helmling,
  %homepage     = www.google.de,
  sheetnumber  = 6,
  %solution, % solution (or solution=true) enables output of solution environments. solution=false (default) disables it,
  % logo = \includegraphics[width=5cm]{image.png}, % can override the standard logo
  % logowidth = 7cm % ... and also its width
}



\usepackage[math,fonts=false]{mh_basic}
\setmathfont{texgyretermes-math.otf}
\setmathfont[range={"29F5}]{XITS Math}
\setmathfont[range={}]{texgyretermes-math.otf}

\newcommand{\mc}{\mathcal}
\newcommand{\GF}{\operatorname{GF}}

\begin{document}
\maketitle
% 
%  \begin{exercise}[title = Implementierung ElGamal-Kryptosystem]
%   Implementieren Sie das ElGamal-Kryptosystem. Gehen Sie davon aus, dass der öffentliche Schlüssel $(q, g, g^a \mod q) \in \mc K_E$ und der zugehörige private Schlüssel $(q, g, a) \in \mc K_D$ gegeben sind.
%  \end{exercise}

 \begin{exercise}[title = ElGamal-Kryptosystem]
  \begin{enumerate}
   \item Alice erhält den mit dem ElGamal-Kryptosystem verschlüsselten Chiffretext $(13, 12) \in \mc C$. Ihr öffentlicher Schlüssel ist $(19, 2, 7) \in \mc K_E$. Wie lautet der Klartext?
   \item Das ElGamal-System funktioniert nicht nur (wie im Skript beschrieben) auf der zyklischen Gruppe $\Z_q^{\times}$ mit $q \in \mathbb P$, sondern auch auf anderen endlichen zyklischen Gruppen. 
   \begin{enumerate}
   \item Beschreiben Sie das ElGamal-System aufbauend auf der zyklische Gruppe $(\Z_n, +)$, $n \in \N$. Zeigen Sie, dass zu jeder Verschlüsselungsfunktion eine Entschlüsselungsfunktion existiert. 
   \item Ist es sinnvoll, das ElGamal-System aufbauend auf der Gruppe $(\Z_n, +)$ zu nutzen?
   \end{enumerate}
%    \item Für das ElGamal-System sei eine Gruppe $\langle g \rangle = \Z_q^{\times}$ mit $q \in \mathbb P$ festgelegt, so dass zur Erzeugung eines Schlüssels $(q, g, g^a \mod q) \in \mc K_E$ nur noch der Exponent $a \in \{2, \dots, q-2\}$ gewählt werden muss. Ist die resultierende Version des ElGamal-Kryptosystems perfekt sicher?
  \end{enumerate}
 \end{exercise}
\begin{solution}
 Der private Schlüssel ist $(19, 2, 6) \in \mc K_D$. Der Klartext ist $8 \in \mc P$. $b$ ist $5$. 
\end{solution}

 
 \begin{exercise}[title = Primzahltests]
  \begin{enumerate}
   \item Bestimmen Sie einen Fermat-Zeugen gegen die Primalität von $21$.
%    \item Zeigen Sie, dass jede Fermat-Zahl $F_n = 2^{2^n} + 1$, $n \in \N$, eine Pseudoprimzahl zur Basis $2$ ist.
   \item Bestimmen Sie alle Miller-Rabin-Zeugen gegen die Primalität von $21$. 
  \end{enumerate}
 \end{exercise}


 \begin{exercise}[title = Primzahlerzeugung]
Entwickeln Sie aufbauend auf dem Miller-Rabin-Test einen probabilistischen Algorithmus zur Erzeugung zufälliger Primzahlen $q \in \mathbb P$. Der Algorithmus soll folgende Eingabe haben:
 \begin{itemize}
 \item Länge der Binärdarstellung $\operatorname{size}(q) \in \N$
 \item obere Schranke $p \in (0, 1]$ für die Fehlerwahrscheinlichkeit, mit der die erzeugte Zahl $q$ keine Primzahl ist 
 \end{itemize}
 Implementieren Sie den entwickelten Algorithmus.
 \end{exercise}

 
\end{document}