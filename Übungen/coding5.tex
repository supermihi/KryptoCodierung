% load document class mhexsheet & select language as option (german or english)
\documentclass[german]{mhexsheet}

% set the parameters for the current exercise sheet with \exerciseSetup (key-value interface)
\exerciseSetup{
  lecture      = Einführung in die Informations- und Codierungstheorie,
  lectureshort = Codierungstheorie, % optional short name for the footer
  semester     = SS 2015,
  deadline     = {6.7.2015},
  lecturer     = Michael Helmling,
  operator     = Carolin Torchiani,
  homepage     = {http://userpages.uni-koblenz.de/~helmling/coding},
  sheetnumber  = 11,
  solution=false,
  %solution, % solution (or solution=true) enables output of solution environments. solution=false (default) disables it,
  % logo = \includegraphics[width=5cm]{image.png}, % can override the standard logo
  % logowidth = 7cm % ... and also its width
}
\usepackage{cleveref}
\usepackage[
  binary-units=true,
  per-mode=symbol,
  exponent-product=⋅,
  locale=DE]{siunitx}
\usepackage{amssymb}
\usepackage[math,fonts=false]{mh_basic}
% == local commands 
\setcounter{MaxMatrixCols}{20}
\newcommand\A{\mathcal A}
\newcommand\B{\operatorname{B}}
\renewcommand\d{\operatorname{d}}
\DeclareMathOperator{\w}{w}
\newcommand{\Mat}{\operatorname{Mat}}
\renewcommand{\t}[1]{\texttt{#1}}
\renewcommand{\(}{\left(}
\renewcommand{\)}{\right)}
\renewcommand\P{\mathcal P}
\newcommand\pErr{p_\mathrm{Err}}
\newcommand\D{D_{\mathrm{KL}}}

\newcommand{\length}{l}
\newcommand{\mc}{\mathcal}
\newcommand{\mb}{\mathbf}
\newcommand{\ol}[1]{[#1]}
\newcommand{\ov}[1]{\overline{#1}}
\newcommand{\id}{\operatorname{id}}
\newcommand{\Sy}{\mathbb S}
\newcommand{\GF}{\operatorname{GF}}
\newcommand{\Ce}{\mathbb C}
\newcommand{\Primes}{\mathbb P}
\newcommand{\Nk}{\operatorname {Nk}}
\newcommand{\Nr}{\operatorname {Nr}}
\newcommand{\AES}{\mathrm{AES}}
\newcommand{\ML}{\mathrm{ML}}
\newcommand{\expand}{\texttt{KeyExpansion}}
\newcommand{\size}{\operatorname{size}}
\newcommand{\parf}{\operatorname{par}}
\newcommand{\im}{\operatorname{im}}
\newcommand{\K}{\mathcal K}
\DeclareMathOperator\Pot{Pot}
\usepackage{tikz}
\renewcommand\C{\mathcal C}
\tikzset{
  extend/.style={shorten >=-#1,shorten <=-#1},
  extend/.default=5mm,
  shorten/.style={shorten >=#1,shorten <=#1},
  shorten/.default=5mm,
  wobbly/.style={decorate,decoration={snake,segment length=15pt,amplitude=0.5pt}},
  brace/.style={decorate,decoration={brace,#1}},
  box/.style={draw,fill=#1,rounded corners,minimum height=15pt},
  circ/.style={circle,inner sep=0mm,minimum size=2mm,draw},
  dot/.style={circle,inner sep=0mm,minimum size=2mm,fill}
}
% ==================

\setmathfont{texgyretermes-math.otf}
\setmathfont[range={"29F5}]{XITS Math}
\setmathfont[range={}]{texgyretermes-math.otf}
\usepackage{booktabs}
\newcommand{\mc}{\mathcal}
\newcommand{\GF}{\operatorname{GF}}
\begin{document}
\maketitle
\begin{exercise}[title=Typizität]
  \begin{enumerate}
    \item Bestätigen Sie die Äquivalenz in Definition 5.50: Ist $X$ eine Zufallsvariable mit Alphabet $A$, $N∈ℕ$, $x∈A^N$ und $β>0$, gilt
  \[
    2^{-N(H(X)+β)} ≤ P_{X^N}(x) ≤ 2^{-N(H(X)-β)} ⇔ \abs{-\frac1N \log P_{X^N}(x) - H(X)} ≤ β \tp
  \]
  \item Zeigen Sie Aussagen 1 und 2 von Lemma 5.55. \emph{Hinweis zu Teil ~1: Schreiben Sie $H(X)$ und $H(Y)$ als Erwartungswert einer von  $(X,Y)$ abgeleiteten Zufallsvariable und verfahren Sie analog zu Lemma~5.52.}
  \end{enumerate}
\end{exercise}
\begin{solution}
  \begin{enumerate}
    \item Einfach umrechnen.
    \item Von Lemma~5.52 erhalten wir
  \[P\( (X,Y)^N ∈ T_{N,β}(X,Y) \) ≥ 1 - ε/3\]
  für hinreichend große $N$. Schreiben wir nach Bemerkung~5.27 $H(X) = \E[f(X,Y)]$ als Erwartungswert der Zufallsvariable $f(X,Y)$ mit $f(x,y) = -\log P_X(x)$, können wir analog zum Beweis von Lemma~5.52 das Gesetz der großen Zahlen auf $f(X,Y)$ anwenden und erhalten $P\(X^N ∈ T_{N,β}(X) \) ≥ 1-ε/3$ und analog $P\(Y^N∈T_{N,β}(Y)\) ≥ 1-ε/3$, jeweils für große $N$. Insgesamt gibt es also ein $N_0$, so dass
      \begin{align*}
        P((X,Y)^N \notin T'_{N,β}) &= P\( (X,Y)^N \notin T_{N,β}(X,Y)\text{ und } X^N \notin T_{N,β}(X) \text{ und } Y^N \notin T_{N,β}(Y)\) \\
        &≤P\( (X,Y)^N \notin T_{N,β}(X,Y)\)+P\(X^N \notin T_{N,β}(X)\)+P\(Y^N \notin T_{N,β}(Y)\)\\
        &≤\frac ε3 + \frac ε3 + \frac ε3 = ε\tk
      \end{align*}
      und damit $P((X,Y)^N ∈ T'_{N,β}) = 1 - P((X,Y)^N \notin T'_{N,β}) ≥ 1 -ε$ für alle $N>N_0$.
      \item Folgt direkt aus Lemma~5.52, da $T'_{N,β}(X,Y) ⊆ T_{N,β}(X,Y)$ und damit $\abs{T'_{N,β}(X,Y)} ≤ \abs{T_{N,β}(X,Y)}$ für $N>N_0$ gilt.
  \end{enumerate}
\end{solution}

\begin{exercise}[title=Quellencodierung mit Blockcodes]
  In dieser Aufgabe werden wir die Aussage des Quellencodierungssatzes – Codierung einer Quelle $(A,X)$ mit durchschnittlich $<H(X)+ε$ Bits pro Symbol – für Blockcodes statt Symbolcodes zeigen. Wir nutzen dazu allgemeinere $(N,K)$-Blockcodes, die von $A^K$ nach $\GF(2)^N$ abbilden. Dazu müssen wir allerdings die Voraussetzungen ein wenig verändern, denn:
  \begin{enumerate}
    \item Sei $(A,X)$ eine Quelle, die mittels eines $(N,K)$-Blockcodes so codiert wird, dass das Ergebnis garantiert und eindeutig decodiert werden kann. Dann gilt $N/K≥\log \abs A$: pro Quellen-Symbol werden mindestens $\log \abs A$ Bits benötigt.
  \end{enumerate}
  Da im Allgemeinen $H(X)<\log \abs A$ gilt, erreichen wir damit nicht dieselbe Datenkompression wie mit Symbolcodes. Wir erlauben deshalb einen kleinen Fehler, der aber beliebig klein werden können muss:
  \begin{enumerate}[resume]
    \item Zeigen Sie: Ist $(A,X)$ eine Quelle und $ε, δ>0$, so gibt es einen $(N,K)$-Blockcode mit $N/K< H(X)+ε$, der die Daten der Quelle so codiert, dass jedes Symbol mit Wahrscheinlichkeit $>1-δ$ korrekt decodiert werden kann. \emph{Hinweis: codieren Sie nur $T_{K,β}(X)$!}
  \end{enumerate}
  Davon ausgehend erhalten wir dann doch wieder ein fehlerfreies Verfahren, indem wir die nicht-typischen $K$-Wörter aus der Quelle durch ein spezielles Codewort $\t s ∈ \GF(2)^N$ \enquote{ankündigen} und danach gesondert übertragen.
  \begin{enumerate}[resume]
    \item Zeigen Sie, dass für jedes $ε>0$ solch ein eindeutig decodierbarer \enquote{Fast-Blockcode} existiert, der im Mittel $<H(X)+ε$ Bits pro Symbol benötigt.
  \end{enumerate}
\end{exercise}
 
\end{document}