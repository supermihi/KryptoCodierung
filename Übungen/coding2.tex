% load document class mhexsheet & select language as option (german or english)
\documentclass[german]{mhexsheet}

% set the parameters for the current exercise sheet with \exerciseSetup (key-value interface)
\exerciseSetup{
  lecture      = Einführung in die Informations- und Codierungstheorie,
  lectureshort = Codierungstheorie, % optional short name for the footer
  semester     = SS 2015,
  deadline     = {15.6.2015},
  %date  = {9.6.2015},
  lecturer     = Michael Helmling,
  operator     = Carolin Torchiani,
  homepage     = {http://userpages.uni-koblenz.de/~helmling/coding},
  sheetnumber  = 8,
  %solution, % solution (or solution=true) enables output of solution environments. solution=false (default) disables it,
  % logo = \includegraphics[width=5cm]{image.png}, % can override the standard logo
  % logowidth = 7cm % ... and also its width
}
\usepackage{cleveref}
\usepackage[
  binary-units=true,
  per-mode=symbol,
  exponent-product=⋅,
  locale=DE]{siunitx}

% == local commands 
\setcounter{MaxMatrixCols}{20}
\newcommand\A{\mathcal A}
\newcommand\B{\operatorname{B}}
\renewcommand\d{\operatorname{d}}
\DeclareMathOperator{\w}{w}
\newcommand{\Mat}{\operatorname{Mat}}
\renewcommand{\t}[1]{\texttt{#1}}
\renewcommand{\(}{\left(}
\renewcommand{\)}{\right)}
\renewcommand\P{\mathcal P}
\newcommand\pErr{p_\mathrm{Err}}
\newcommand\D{D_{\mathrm{KL}}}

\newcommand{\length}{l}
\newcommand{\mc}{\mathcal}
\newcommand{\mb}{\mathbf}
\newcommand{\ol}[1]{[#1]}
\newcommand{\ov}[1]{\overline{#1}}
\newcommand{\id}{\operatorname{id}}
\newcommand{\Sy}{\mathbb S}
\newcommand{\GF}{\operatorname{GF}}
\newcommand{\Ce}{\mathbb C}
\newcommand{\Primes}{\mathbb P}
\newcommand{\Nk}{\operatorname {Nk}}
\newcommand{\Nr}{\operatorname {Nr}}
\newcommand{\AES}{\mathrm{AES}}
\newcommand{\ML}{\mathrm{ML}}
\newcommand{\expand}{\texttt{KeyExpansion}}
\newcommand{\size}{\operatorname{size}}
\newcommand{\parf}{\operatorname{par}}
\newcommand{\im}{\operatorname{im}}
\newcommand{\K}{\mathcal K}
\DeclareMathOperator\Pot{Pot}
\usepackage{tikz}
\renewcommand\C{\mathcal C}
\tikzset{
  extend/.style={shorten >=-#1,shorten <=-#1},
  extend/.default=5mm,
  shorten/.style={shorten >=#1,shorten <=#1},
  shorten/.default=5mm,
  wobbly/.style={decorate,decoration={snake,segment length=15pt,amplitude=0.5pt}},
  brace/.style={decorate,decoration={brace,#1}},
  box/.style={draw,fill=#1,rounded corners,minimum height=15pt},
  circ/.style={circle,inner sep=0mm,minimum size=2mm,draw},
  dot/.style={circle,inner sep=0mm,minimum size=2mm,fill}
}
% ==================

\usepackage[math,fonts=false]{mh_basic}
\setmathfont{texgyretermes-math.otf}
\setmathfont[range={"29F5}]{XITS Math}
\setmathfont[range={}]{texgyretermes-math.otf}

\newcommand{\mc}{\mathcal}
\newcommand{\GF}{\operatorname{GF}}

\begin{document}
\maketitle

\begin{exercise}
  \begin{enumerate}
    \item Sei $X$ eine Zufallsvariable mit Alphabet $A=\{a,b,c\}$ und $p_a = \num{.1}$, $p_b = \num{.2}$, $p_c=\num{.7}$. Berechnen Sie $P( P_X(X) ∈ [\num{.15}, \num{.5}])$ und
    \[P\(\abs{\log\frac{P_X(X)}{\num{.2}}} > \num{.05}\)\tp\]
    \item Seien $X, A$ wie oben und $f\colon A→ℝ$ gegeben durch $f(a)=10$, $f(b)=5$, $f(c)=10/7$. Berechnen Sie $\E[f(X)]$ und $\E[1/P_X(X)]$.
    \item Sei $X$ eine beliebige Zufallsvariable mit Alphabet $A$. Was ist $\E[1/P_X(X)]$?
  \end{enumerate}
\end{exercise}

\begin{exercise}[title=Zipf'sches Gesetz]
  Seien $a_1,\dotsc, a_n$ die Buchstaben einer Sprache (\zB Deutsch), absteigend sortiert nach der Häufigkeit ihres Auftretens. Das \emph{Zipf'sche Gesetz} besagt, dass die Häufigkeit von $a_i$ ungefähr proportional zu $1/i$ ist.
  \begin{enumerate}
    \item Stellen Sie eine Formel für die Häufigkeit von $a_i$ auf.
    \item Berechnen Sie die Entropie eines zufälligen Buchstabens der deutschen Sprache ausgehend von der Annahme, dass das Zipf'sche Gesetz exakt zutrifft. Berücksichtigen Sie dabei nur die $26$ Buchstaben des lateinischen Alphabets.
    \item Das Gesetz gilt auch für die \emph{Wörter} einer Sprache. Berechnen Sie die Entropie eines zufälligen deutschen Wortes. Gehen Sie von etwa \num{500000} Wörtern aus und nehmen Sie einen Computer zu Hilfe!
    \item Ein deutsches Wort ist im Schnitt etwa \num{5,3} Buchstaben lang. Um wie viel kürzer könnte es bei Shannon-Codierung sein?
  \end{enumerate}
\end{exercise}

\begin{exercise}
  Zeigen Sie durch Konstruktion passender Beispiele, dass die beiden Abschätzungen in Satz~5.17 scharf sind:
  \begin{enumerate}
    \item Es gibt eine Quelle $(A,X)$ und einen passenden Präfixcode $E$, so dass $H(X) = l_E$ gilt.
    \item Für jedes $ε>0$ gibt es eine Quelle $(A,X)$ und einen Shannon-Code $E$, so dass $l_E ≥ H(X) + 1 - ε$ gilt.
  \end{enumerate}
\end{exercise}
\end{document}