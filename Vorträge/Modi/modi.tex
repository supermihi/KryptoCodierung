\documentclass{beamer}

\usepackage[utf8]{inputenc}
\usepackage[ngerman]{babel}

\usepackage{array}
\usepackage{calc}
\usepackage{graphicx}
\usepackage{amssymb}
\usepackage{amsthm}
\usepackage{mathtools}
% \usepackage{xcolor}

\usepackage{algorithmicx}

\usepackage{algorithm}
\usepackage{algpseudocode}

\renewcommand{\algorithmicrequire}{\textbf{Input:}}
\renewcommand{\algorithmicensure}{\textbf{Output:}}


\usepackage{textpos}
\usepackage{color}
\usepackage{tikz}
\usetikzlibrary{shapes, arrows, positioning, decorations.pathmorphing}

\setbeamertemplate{bibliography item}[triangle]

\renewcommand{\atop}[2]{\genfrac{}{}{0pt}{2}{#1}{#2}}

\setbeamerfont{quote}{shape=\upshape,family=\rmfamily}

\newcommand{\mc}{\mathcal}
\newcommand{\mb}{\mathbf}
\newcommand{\N}{\mathbb{N}}
\newcommand{\Z}{\mathbb{Z}}
\newcommand{\Q}{\mathbb{Q}}
\newcommand{\R}{\mathbb{R}}
\newcommand{\Sy}{\mathcal{S}}
\newcommand{\tb}[1]{{\textcolor{blue}{#1}}}
\newcommand{\tred}[1]{{\textcolor{red}{#1}}}
\newcommand{\Nk}{\operatorname {Nk}}
\newcommand{\Nr}{\operatorname {Nr}}
\newcommand{\expand}{\texttt{KeyExpansion}}
\newcommand{\size}{\operatorname{size}}
\newcommand{\parf}{\operatorname{par}}
\newcommand{\GF}{\operatorname{GF}}
\newcommand{\ol}[1]{[#1]}

\theoremstyle{plain}
\newtheorem{thm}{Satz}[section]
\newtheorem{lem}[thm]{Lemma}
\newtheorem{prop}[thm]{Proposition}
\newtheorem{defn}[thm]{Definition}
\newtheorem{expl}[thm]{Beispiel}
\newtheorem{rem}[thm]{Bemerkung}
\newtheorem{cor}[thm]{Korollar}
\newtheorem{notn}[thm]{Notation}

\usetheme{Boadilla}
\setbeamertemplate{navigation symbols}{} %Leave out navigation symbols
 \setbeamertemplate{footline}
        {
      \leavevmode%
      \hbox{%
      \begin{beamercolorbox}[wd=.333333\paperwidth,ht=2.25ex,dp=1ex,center]{author in head/foot}%
        \usebeamerfont{author in head/foot}\insertshortauthor~~(\insertshortinstitute)
      \end{beamercolorbox}%
      \begin{beamercolorbox}[wd=.333333\paperwidth,ht=2.25ex,dp=1ex,center]{title in head/foot}%
        \usebeamerfont{title in head/foot}\insertshorttitle
      \end{beamercolorbox}%
      \begin{beamercolorbox}[wd=.333333\paperwidth,ht=2.25ex,dp=1ex,right]{date in head/foot}%
        \usebeamerfont{date in head/foot}\insertshortdate{}\hspace*{2em}

    %#turning the next line into a comment, erases the frame numbers
        %\insertframenumber{} / \inserttotalframenumber\hspace*{2ex} 

      \end{beamercolorbox}}%
      \vskip0pt%
    }
    
    
\title[Verschlüsselungsmodi]{Verschlüsselungsmodi}
\subtitle{Von Blockchiffren auf $A^n$ zu Kryptosystemen auf $A^*$}
\author[Torchiani]{Carolin Torchiani}
\institute[Uni Koblenz]{\vspace{-0.5cm}\begin{center}
 \includegraphics[width=27mm]{logo_uni-koblenz}
\end{center}
}

\date{Koblenz, 4.\ Mai 2015}


\begin{document}

\frame{\titlepage}

\addtobeamertemplate{frametitle}{~}{%
\begin{textblock*}{40mm}[0.75, 0.75](\textwidth, 0\textheight)
\includegraphics[width=27mm]{logo_uni-koblenz}
\end{textblock*}}

\tikzset{pblock/.style = {draw, rectangle split, rectangle split vertical,
                      rectangle split parts=2, very thick, fill=blue!20, rounded corners}}

\begin{frame}{Verschlüsselungsmodi}{\ }
\begin{block}{ }
Sei $A$ ein Alphabet. Wie konstruiert man aus einer Blockchiffre mit Klartext- und Chiffretextraum $A^n$ ein Kryptosystem mit Klartext- und Chiffretextraum $A^*$?
\end{block}
\end{frame}

\begin{frame}{Beispiel}{One-Time-Pad}
\only<1>{
\begin{block}{Vom One-Time-Pad auf $\Z_2^4$ zum Kryptosystem auf $\Z_2^*$}
 \textbf{Gegeben:}
 
 Alphabet \textcolor{orange}{$\Z_2$}, $\mc P = \mc K = \mc C = \tred{\Z_2^4}$, $E_e(p) = p+e \in \Z_2^4$ für alle $p, e \in \Z_2^4$
 \medskip
 
 z.\,B.\ $E_{(\ol 1, \ol 0, \ol 1, \ol 1)}(\ol 0, \ol 0, \ol 1, \ol 1) = ([1], [0], [0], [0])$
 

 \bigskip
 \textbf{Gesucht:} 
 
 Kryptosystem mit Alphabet $\tred{\Z_2^4}$ und Klar-, Schlüsseltextraum \textcolor{blue}{$(\Z_2^4)^*$}
 \medskip
 
 z.\,B.\ $E_{(\ol 1, \ol 0, \ol 1, \ol 1)}((\ol 0, \ol 0, \ol 1, \ol 1), (\ol 0, \ol 0, \ol 1, \ol 1), (\ol 0, \ol 0, \ol 1, \ol 1)) \tred{\; = \; ???}$
\end{block}}

\only<2>{
\begin{block}{One-Time-Pad}
 \textbf{Gegeben:}
 
 Alphabet \textcolor{orange}{$\Z_2$}, $\mc P = \mc K = \mc C = \tred{\Z_2^4}$, $E_e(p) = p+e \in \Z_2^4$ für alle $p, e \in \Z_2^4$
 \medskip
 
 z.\,B.\ $E_{(\ol 1, \ol 0, \ol 1, \ol 1)}(\ol 0, \ol 0, \ol 1, \ol 1) = ([1], [0], [0], [0])$
 

 \bigskip
 \textbf{Gesucht:} 
 
 Kryptosystem mit Alphabet $\tred{\Z_2^4}$ und Klar-, Schlüsseltextraum \textcolor{blue}{$(\Z_2^4)^*$}
 \medskip
 
 z.\,B.\ $E_{(\ol 1, \ol 0, \ol 1, \ol 1)}((\ol 0, \ol 0, \ol 1, \ol 1,\; \ol 0, \ol 0, \ol 1, \ol 1, \; \ol 0, \ol 0, \ol 1, \ol 1)) \tred{\; = \; ???}$
\end{block}}
\end{frame}

\begin{frame}{Verschlüsselungsmodi}{\ }
 
 \begin{block}{Input}
  Blockchiffre auf dem Alphabet $A = \Z_m$ mit Blocklänge $n \in \N$, d.\,h.\ der \tred{Klar- und Chiffretextraum ist $\Z_m^n$}
 \end{block}

 \begin{block}{Output}
 Kryptosystem mit \tred{Alphabet $B = \Z_m^n$} und Klar- und Chiffretextraum $(\Z_m^n)^*$
 \end{block}
\pause

\begin{block}{Notation}
 \begin{itemize}
  \item $e \in \mc K$ ein Schlüssel aus dem Schlüsselraum der Blockchiffre
  \item $p_1p_2\dotsc \in \mc P = B^*$ ein Wort, d.\,h.\ $p_i \in B = \Z_m^n$ für alle $i \geq 1$
  \item \textbf{Konstruktion:} 
  
  $E_e(p_1p_2p_3\dotsc) = c_1c_2c_3\dotsc \in \mc C = B^*$ mit $c_i \in B = \Z_m^n$ für alle $i > 1$ 
 \end{itemize}
\end{block}
\end{frame}

\begin{frame}{ECB-Mode}{Verschlüsselungsmodi}
  \begin{footnotesize}
 \begin{block}{ECB-Mode (electronic codebook)}
  $c_i := E_e(p_i)$ für alle $i \geq 1$
 \end{block}
 \begin{center}
   \begin{tikzpicture}[scale = 0.7]
    \tikzstyle{opmode} = [draw, rectangle, minimum width = 1cm]
    \tikzstyle{la} = [-latex]
    \tikzstyle{label} = [fill=white]
    \node[opmode] (pi-1) at (-3, 1) {\footnotesize{$p_{i-1}$}};
    \node[opmode] (pi) at (0, 1) {\footnotesize{$p_{i}$}};
    \node[opmode] (pi+1) at (3, 1) {\footnotesize{$p_{i+1}$}};
    \node[opmode] (ci-1) at (-3, -1) {\footnotesize{$c_{i-1}$}};
    \node[opmode] (ci) at (0, -1) {\footnotesize{$c_{i}$}};
    \node[opmode] (ci+1) at (3, -1) {\footnotesize{$c_{i+1}$}};
    \draw[la] (pi-1) -- node[label] {\scriptsize{Schlüssel $e$}} (ci-1);
    \draw[la] (pi) -- node[label] {\scriptsize{Schlüssel $e$}} (ci);
    \draw[la] (pi+1) -- node[label] {\scriptsize{Schlüssel $e$}} (ci+1);
   \end{tikzpicture}
 \end{center} 
\pause
  \begin{block}{ One-Time-Pad auf $\Z_{26}^3$ als Blockchiffre}
 \[e = ([10], [8], [4]) \in \Z_{26}^3\]
 \[p = ([12], [7], [23], \quad [2], [17], [4], \quad [1], [9], [13]) \in \Z_{26}^9\]
   \begin{align*} E_e(p) = &([12 +10], [7+8], [23+4], \; [2+10], [17+8], [4+4], \; [1+10], [9+8], [13+4]) \\
   = & ([22], [15], [1], \quad [12], [25], [8], \quad [11], [17], [17]) \in \Z_{26}^9
   \end{align*}
   \end{block}
   \end{footnotesize}
\end{frame}

\begin{frame}{CBC-Mode}{Verschlüsselungsmodi}
\begin{footnotesize}
 \begin{block}{CBC-Mode (cipher-block chaining)}
 $c_i := E_e(p_i + c_{i-1})$ für alle $i \geq 1$ mit $c_0 \in B$ fest und öffentlich gewählt
 \end{block}
\begin{center}
   \begin{tikzpicture}[scale=0.6]
    \tikzstyle{opmode} = [draw, rectangle, minimum width = 1cm]
    \tikzstyle{la} = [-latex]
    \tikzstyle{label} = [fill=white]
    \node[opmode] (pi-1) at (-4, 1) {\footnotesize{$p_{i-1}$}};
    \node[opmode] (pi) at (0, 1) {\footnotesize{$p_{i}$}};
    \node[opmode] (pi+1) at (4, 1) {\footnotesize{$p_{i+1}$}};
    \node[opmode] (ci-1) at (-4, -2) {\footnotesize{$c_{i-1}$}};
    \node[opmode] (ci) at (0, -2) {\footnotesize{$c_{i}$}};
    \node[opmode] (ci+1) at (4, -2) {\footnotesize{$c_{i+1}$}};
    \node (oi-1) at (-4, 0) {\small{$\oplus$}};
    \node (oi) at (0, 0) {\small{$\oplus$}};
    \node (oi+1) at (4, 0) {\small{$\oplus$}};
%     \draw (-4, 0) circle (5pt);
%     \draw (0, 0) circle (5pt);
%     \draw (4, 0) circle (5pt);
    \draw[la, shorten >= -3pt] (pi-1) -- (oi-1);
    \draw[la, shorten >= -3pt] (pi) -- (oi);
    \draw[la, shorten >= -3pt] (pi+1) -- (oi+1);
    \draw[la, shorten <= -3pt] (oi-1) -- node[label]  {\scriptsize{Schlüssel $e$}} (ci-1);
    \draw[la, shorten <= -3pt] (oi) -- node[label]  {\scriptsize{Schlüssel $e$}} (ci);
    \draw[la, shorten <= -3pt] (oi+1) -- node[label]  {\scriptsize{Schlüssel $e$}} (ci+1);
    \draw[la, shorten >= -3pt] (-7, -2) -- (-6, -2) -- (-6, 0) -- (oi-1);
    \draw[la, shorten >= -3pt] (ci-1) -- (-2, -2) -- (-2, 0) -- (oi);
    \draw[la, shorten >= -3pt] (ci) -- (2, -2) -- (2, 0) -- (oi+1);
    \draw[la] (ci+1) -- (6, -2);
   \end{tikzpicture}
  \end{center}
  \pause
  \vspace{-3pt}
  \begin{block}{One-Time-Pad auf $\Z_{26}^3$ als Blockchiffre}
 \[e = ([10], [8], [4]) \in \Z_{26}^3, \; c_0 = ([1], [1], [1])\]
 \[p = ([12], [7], [23], \quad [2], [17], [4], \quad [1], [9], [13]) \in \Z_{26}^9\]
   \begin{align*} c_1 = E_e(p_1 + c_0) = \;&[([12], [7], [23]) + ([1], [1], [1])] + ([10], [8], [4])& & = ([23], [16], [2]) \\
   c_2 =  E_e(p_2 + c_1) = \; & [([2], [17], [4]) +  ([23], [16], [2])] + ([10], [8], [4])&& = ([9], [15], [10])\\
   c_3 = E_e(p_3 + c_2) = \; & [([1], [9], [13]) +  ([9], [15], [10])] + ([10], [8], [4])&& = ([20], [6], [1]) 
   \end{align*}
   \end{block}
   \end{footnotesize}
\end{frame}

\begin{frame}{CFB-Mode}{Verschlüsselungsmodi}
\begin{footnotesize}
  \begin{block}{CFB-Mode (cipher feedback)}
 $c_i := p_i + E_e(c_{i-1})$ für alle $i \geq 1$ mit $c_0 \in B$ fest und öffentlich gewählt
 \end{block}
    \begin{center}
   \begin{tikzpicture}[scale=0.6]
    \tikzstyle{opmode} = [draw, rectangle, minimum width = 1cm]
    \tikzstyle{la} = [-latex]
    \tikzstyle{label} = [fill=white]
        \tikzstyle{label} = [fill=white]
    \node[opmode] (pi-1) at (-4, 1) {$p_{i-1}$};
    \node[opmode] (pi) at (0, 1) {$p_{i}$};
    \node[opmode] (pi+1) at (4, 1) {$p_{i+1}$};
    \node[opmode] (ci-1) at (-4, -2) {$c_{i-1}$};
    \node[opmode] (ci) at (0, -2) {$c_{i}$};
    \node[opmode] (ci+1) at (4, -2) {$c_{i+1}$};
    \node (oi-1) at (-4, 0) {\small{$\oplus$}};
    \node (oi) at (0, 0) {\small{$\oplus$}};
    \node (oi+1) at (4, 0) {\small{$\oplus$}};
    \draw[la, shorten >= -3pt] (pi-1) -- (oi-1);
    \draw[la, shorten >= -3pt] (pi) -- (oi);
    \draw[la, shorten >= -3pt] (pi+1) -- (oi+1);
    \draw[la, shorten <= -3pt] (oi-1) -- (ci-1);
    \draw[la, shorten <= -3pt] (oi) --  (ci);
    \draw[la, shorten <= -3pt] (oi+1) -- (ci+1);
    \draw[la, shorten >= -3pt] (-7, -2) -- (-6, -2) -- node[label] {\scriptsize{Schlüssel $e$}} (-6, 0) -- (oi-1);
    \draw[la, shorten >= -3pt] (ci-1) -- (-2, -2) -- node[label] {\scriptsize{Schlüssel $e$}} (-2, 0) -- (oi);
    \draw[la, shorten >= -3pt] (ci) -- (2, -2) -- node[label] {\scriptsize{Schlüssel $e$}} (2, 0) -- (oi+1);
    \draw[la] (ci+1) -- (6, -2);
   \end{tikzpicture}
  \end{center}
  \pause
    \vspace{-3pt}
    \begin{block}{One-Time-Pad auf $\Z_{26}^3$ als Blockchiffre}
     \[e = ([10], [8], [4]) \in \Z_{26}^3, \; c_0 = ([1], [1], [1])\]
 \[p = ([12], [7], [23], \quad [2], [17], [4], \quad [1], [9], [13]) \in \Z_{26}^9\]
   \begin{align*}
   c_1 = \;& p_1 + E_e(c_{0}) = ([12], [7], [23]) + [([1], [1], [1]) + [10], [8], [4])] &&= ([23], [16], [2]), \\
   c_2 = \;& p_2 + E_e(c_1) = ([2], [17], [4]) + [([23], [16], [2]) + [10], [8], [4])] &&= ([9], [15], [10]), \\
   c_3 = \;& p_3 + E_e(c_2) = ([1], [9], [13]) + [([9],[15], [10]) + [10], [8], [4])] &&= ([20], [6], [1]).
   \end{align*}
   \end{block}
   \end{footnotesize}
  \end{frame}
  
  \begin{frame}{OFB-Mode}{Verschlüsselungsmodi}
\begin{footnotesize}
  \begin{block}{OFB-Mode (output feedback)}
$c_i := p_i + s_i$ für alle $i \geq 1$ mit $s_{0} = c_{0}$ und $s_i = E_e(s_{i-1})$ für alle $i \geq 1$   
 \end{block}
  \begin{center}
   \begin{tikzpicture}[scale=0.6]
    \tikzstyle{opmode} = [draw, rectangle, minimum width = 1cm]
    \tikzstyle{la} = [-latex]
        \tikzstyle{label} = [fill=white]
        \tikzstyle{label} = [fill=white]
    \node[opmode] (si-1) at (-5, 1) {$s_{i-1}$};
    \node[opmode] (si) at (0, 1) {$s_i$};
    \node[opmode] (si+1) at (5, 1) {$s_{i+1}$};
    \node[opmode] (pi-1) at (-7.5, 0) {$p_{i-1}$};
    \node[opmode] (pi) at (-2.5, 0) {$p_{i}$};
    \node[opmode] (pi+1) at (2.5, 0) {$p_{i}$};
    \node[opmode] (ci-1) at (-5, -1) {$c_{i-1}$};
    \node[opmode] (ci) at (0, -1) {$c_{i}$};
    \node[opmode] (ci+1) at (5, -1) {$c_{i+1}$};
    \node (oi-1) at (-5, 0) {\small{$\oplus$}};
    \node (oi) at (0, 0) {\small{$\oplus$}};
    \node (oi+1) at (5, 0) {\small{$\oplus$}};
    \draw[la, shorten >= -3pt] (pi-1) -- (oi-1);
    \draw[la, shorten >= -3pt] (pi) -- (oi);
    \draw[la, shorten >= -3pt] (pi+1) -- (oi+1);
    \draw[la, shorten >= -3pt] (si-1) -- (oi-1);
    \draw[la, shorten >= -3pt] (si) -- (oi);
    \draw[la, shorten >= -3pt] (si+1) -- (oi+1);
    \draw[la, shorten <= -3pt] (oi-1) -- (ci-1);
    \draw[la, shorten <= -3pt] (oi) --  (ci);
    \draw[la, shorten <= -3pt] (oi+1) -- (ci+1);
    \draw[la] (-9.5, 1) -- node [label] {\scriptsize{Schlüssel $e$}} (si-1);
    \draw[la] (si-1) -- node [label] {\scriptsize{Schlüssel $e$}} (si);
    \draw[la] (si) -- node [label] {\scriptsize{Schlüssel $e$}} (si+1);
    \draw[la] (si+1) -- (6.5, 1);
     \end{tikzpicture}
\end{center}
\pause
    \begin{block}{One-Time-Pad auf $\Z_{26}^3$ als Blockchiffre}
     \[e = ([10], [8], [4]) \in \Z_{26}^3, \; c_0 = s_0 = ([1], [1], [1])\]
 \[p = ([12], [7], [23], \quad [2], [17], [4], \quad [1], [9], [13]) \in \Z_{26}^9\]
   \begin{align*}
   c_1 = \;& p_1 + E_e(s_{0}) = ([12], [7], [23]) + [([1], [1], [1]) + [10], [8], [4])] &&= ([23], [16], [2]), \\
   c_2 = \;& p_2 + E_e(s_1) = ([2], [17], [4]) + [([11], [9], [5]) + ([10], [8], [4])] &&= ([23], [8], [13]), \\
   c_3 = \;& p_3 + E_e(s_2) = ([1], [9], [13]) + [([21],[17], [9]) + ([10], [8], [4])] &&= ([6], [8], [0]).
   \end{align*}
   \end{block}
   \end{footnotesize}
  \end{frame}
  
  \begin{frame}{Verschlüsselungsmodi}{\ }
   \begin{center}
   \begin{block}{}
    Von der Blockchiffre auf $A^n$ zum Kryptosystem auf $A^*$. \quad \Huge{\textcolor{green}{\checkmark}}
    \end{block}
   \end{center}

  \end{frame}

\end{document}



